\documentclass{uninove-ppgi} %courier ou times

\begin{document}
\lstset{
    language=xml,
    tabsize=3,
    frame=shadowbox,
    rulesepcolor=\color{gray},
    xleftmargin=20pt,
    framexleftmargin=15pt,
    keywordstyle=\color{blue}\bf,
    commentstyle=\color{OliveGreen},
    stringstyle=\color{red},
    numbers=left,
    numberstyle=\tiny,
    numbersep=5pt,
    breaklines=true,
    showstringspaces=false,
    basicstyle=\footnotesize,
    emph={food,name,price},emphstyle={\color{magenta}}
}

% Posiciona o logo da Uni9 no topo
\includegraphics[height=1.5cm]{uninove-logo}

% parametros de capa e folha de rosto (é necessário configurar todos)
\Universidade{UNIVERSIDADE NOVE DE JULHO - UNINOVE}

\Autor{EXCLUÍDOS OS DADOS SOBRE OS AUTORES EM ATENDIMENTO A LGPD - LEI GERAL DE PROTEÇÃO DE DADOS}

% ATENÇÃO: NÃO INCLUIR NOME E RA DE NENHUM ALUNO
% EM NENHUMA PARTE DO DOCUMENTO

\Titulo{Projeto de Criação do e-commerce}

% Inserir o nome do projeto no formato que está abaixo (Olhe o nome da disciplina na central do aluno)
\Tipoprojeto{Tech Inova Shop)}

% Informe qual o curso: Bacharel ou Tecnólogo + curso
\Curso{Ánalise e Desenvolvimento de Sistemas}

% NÃO ALTERAR
\Orientador{Edson Melo de Souza, Dr.}

% Inserir o ano correspondente
\Ano{2024}

% gera a capa automaticamente
\capa

% gera folha de rosto automaticamente
\folharosto

% ##################### Início dos elementos pré-textuais ############################

% Resumo (Obrigatório)
% !TEX root = ..\main.tex

\PalavrasChave{Tech Inova Shop, e-commerce, tecnologia, plataforma digital, experiência do usuário,}
\centeredchapterstyle
\begin{resumo}
    \noindent\textbf{Contexto}:A Tech Inova Shop, uma empresa de tecnologia, reconhece a necessidade de expandir sua presença no mercado digital através da implantação de um e-commerce. Diante do crescimento contínuo do comércio eletrônico e da demanda dos consumidores por uma experiência de compra online conveniente e eficiente, a empresa busca capitalizar essas oportunidades para impulsionar seu crescimento e aumentar sua participação no mercado. \textbf{Objetivo}:O objetivo deste projeto é desenvolver e implementar um e-commerce para a Tech Inova Shop, proporcionando aos clientes uma plataforma de compras online intuitiva, segura e personalizada. A empresa visa não apenas aumentar suas vendas e alcance, mas também melhorar a experiência do cliente e fortalecer sua posição competitiva no mercado de tecnologia. \textbf{Método}:O projeto será conduzido em várias etapas, começando com uma análise detalhada das necessidades dos clientes e das tendências do mercado. Com base nessa análise, será desenvolvida uma plataforma robusta e escalável, utilizando as melhores práticas de design responsivo e usabilidade. Funcionalidades como recomendação personalizada, chatbots e integração com redes sociais serão cuidadosamente integradas para aumentar a interação dos clientes.

Medidas de segurança serão implementadas para proteger as transações online, incluindo criptografia, prevenção de fraudes e autenticação de dois fatores. Um plano abrangente de marketing digital será elaborado, incluindo estratégias de SEO, publicidade online e uso de redes sociais para aumentar a visibilidade da marca e atrair novos clientes para o e-commerce \textbf{Resultados}: Espera-se que a implementação do e-commerce resulte em um aumento significativo nas vendas e na fidelidade dos clientes. Métricas como taxa de conversão, ticket médio e satisfação do cliente serão monitoradas para avaliar o desempenho da plataforma. O e-commerce da Tech Inova Shop deverá proporcionar uma experiência de compra satisfatória, aumentando assim a fidelidade dos clientes e impulsionando o crescimento do negócio. \textbf{Conclusão}:A implantação bem-sucedida do e-commerce representa um marco importante na estratégia de crescimento da Tech Inova Shop. Ao oferecer uma plataforma de compras online de alta qualidade, a empresa está posicionada para atender às demandas dos clientes modernos e capitalizar as oportunidades de crescimento no mercado digital. Com medidas de segurança robustas e estratégias de marketing eficazes, a Tech Inova Shop está preparada para alcançar seus objetivos e consolidar sua posição como líder no setor de tecnologia.
\end{resumo}

% Abstract (Obrigatório) - resumo em inglês
% !TEX root = ..\main.tex

\KeyWords{Keyword 1, Keyword 2, Keyword 3, Keyword 4, Keyword 5, Keyword 6.}
\centeredchapterstyle
\begin{abstract}
    \noindent\textbf{Contextualization}:Tech Inova Shop, a technology company, recognizes the need to expand its presence in the digital market by implementing an e-commerce platform. Faced with the continuous growth of e-commerce and consumer demand for a convenient and efficient online shopping experience, the company seeks to capitalize on these opportunities to drive its growth and increase its market share. \textbf{Objetive}: The objective of this project is to develop and implement an e-commerce platform for Tech Inova Shop, providing customers with an intuitive, secure, and personalized online shopping platform. The company aims not only to increase sales and reach but also to enhance the customer experience and strengthen its competitive position in the technology market. \textbf{Method}: The project will be conducted in several stages, starting with a detailed analysis of customer needs and market trends. Based on this analysis, a robust and scalable platform will be developed, utilizing best practices in responsive design and usability. Features such as personalized recommendation, chatbots, and integration with social networks will be carefully integrated to increase customer interaction.

Security measures will be implemented to protect online transactions, including encryption, fraud prevention, and two-factor authentication. A comprehensive digital marketing plan will be developed, including SEO strategies, online advertising, and the use of social networks to increase brand visibility and attract new customers to the e-commerce platform. \textbf{Results}:It is expected that the implementation of the e-commerce platform will result in a significant increase in sales and customer loyalty. Metrics such as conversion rate, average order value, and customer satisfaction will be monitored to assess the performance of the platform. Tech Inova Shop's e-commerce platform is expected to provide a satisfactory shopping experience, thereby increasing customer loyalty and driving business growth. \textbf{Conclusion}: The successful implementation of the e-commerce platform represents a significant milestone in Tech Inova Shop's growth strategy. By offering a high-quality online shopping platform, the company is positioned to meet the demands of modern customers and capitalize on growth opportunities in the digital market. With robust security measures and effective marketing strategies, Tech Inova Shop is poised to achieve its goals and consolidate its position as a leader in the technology sector.
\end{abstract}


% Sumário (Obrigatório)
\begingroup
\makeatletter \let\ps@plain\ps@empty \makeatother
\tableofcontents % sumário
\endgroup
\thispagestyle{empty}

% Lista de figuras
\listoffigures
\thispagestyle{empty}


% Lista de abreviaturas (Opcional)
\begin{listaabreviaturas}%
    MM & Morfologia matemática \\
    CC & Componente conexo \\
    EE & Elemento estruturante
\end{listaabreviaturas}

% ############### Fim dos elementos pré-textuais ######################

\regularchapterstyle

% ############### Início dos Capítulos (Obrigatório ) #################

% Introdução
\chapter{Introdução}
\label{ch:introducao}
\begin{resumocapitulo}
	Este capítulo apresenta uma visão geral do projeto de implantação de um e-commerce para a empresa Tech Inova Shop. Inicialmente, é contextualizada a necessidade da empresa de expandir sua presença no mercado digital diante do crescimento contínuo do comércio eletrônico e da demanda dos consumidores por uma experiência de compra online eficiente. Em seguida, são delineados os objetivos do projeto, que visam desenvolver uma plataforma de e-commerce robusta e escalável, com foco na melhoria da experiência do cliente e no aumento das vendas. O método utilizado para alcançar esses objetivos é descrito, incluindo análise de mercado, desenvolvimento de funcionalidades e estratégias de segurança e marketing. Por fim, são apresentadas as expectativas em relação aos resultados do projeto, destacando a importância de métricas como taxa de conversão e satisfação do cliente para avaliar o sucesso da implantação do e-commerce da Tech Inova Shop.
\end{resumocapitulo}

\section{Citações diretas e indiretas}
\label{sec:citacoes}
\subsection{Citação Direta}
\label{subsec:citacao_direta}
"O e-commerce revolucionou a forma como compramos e vendemos produtos e serviços, proporcionando uma experiência de compra mais conveniente, acessível e personalizada para os consumidores." (SILVA, 2020, p. 25).

\subsection{Citação Indireta}
\label{subsec:citacao_indireta}
De acordo com Silva (2020), o e-commerce trouxe consigo uma série de mudanças no cenário do varejo, impactando tanto o comportamento dos consumidores quanto as estratégias das empresas.
\section{Inclusão de Figura}
\label{sec:figura}
A figura 1.1 são diagramas do sistema
\begin{figure}[!ht]
	{\centering
		\caption{Diagrama DER.}
		
  \includegraphics[width=0.9\textwidth]{figuras/dados.png}
  
  \caption{Modelo ER.}
  \includegraphics[width=0.9\textwidth]{figuras/dados1.png}
		\label{fig:identificador_da_figura}
		\fonte{Autor}
	}
\end{figure}



% Base Teórica
\chapter{Fundamentação Teórica}
\label{ch:identificador}
	\begin{resumocapitulo}
		Aqui vai um pequeno resumo do capítulo.
	\end{resumocapitulo}

	\section{Visão Geral}
		Science Direct-Revista científica-ISSN 0346-251X
  Explorando os mecanismos  que impulsionam o crescimento do comércio eletrônico Business-to-Consumer (B2C) na Europa a partir de uma perspectiva teórica de crescimento.
  Exploring the mechanisms driving Business-to-Consumer (B2C) e-commerce growth in Europe from a growth theory perspective.
Explorando los mecanismos que impulsan el crecimiento del comercio electrónico Business-to-Consumer (B2C) en Europa desde una perspectiva teórica de crecimiento.
  

	\section{Conteúdo 1}
	\label{sec:identificao}
        Resumo
        As tecnologias de informação e comunicação (TICs) continuam a ter um profundo efeito nas economias e sociedades onde são utilizadas. Neste artigo, propomos três teorias relacionadas para descrever o mecanismo subjacente ao crescimento das receitas de comércio eletrônico em nível nacional. A teoria do crescimento endógeno postula que os principais impulsionadores do crescimento do comércio eletrônico são internos a um país. A teoria do crescimento exógeno sugere que os principais impulsionadores do crescimento do comércio eletrônico são externos a um sistema econômico e refletem as forças da economia regional. Uma combinação dessas teorias, uma teoria mista de crescimento endógeno-exógeno, incorpora impulsionadores tanto da economia quanto da região de um país. Testamos várias hipóteses sobre o crescimento do comércio eletrônico no contexto dessas teorias. As variáveis-chave incluem penetração da Internet, intensidade do investimento em telecomunicações, disponibilidade de capital de risco e cartão de crédito e nível de educação. Os dados são retirados de 17 países europeus ao longo de um período de cinco anos, de 2000 a 2004, e são analisados usando regressão de dados em painel com termos de erro robustos, uma variante de mínimos quadrados ponderados. Os resultados mostram a eficácia diferencial de impulsionadores internos e externos como precursores endógenos e exógenos do crescimento do comércio eletrônico entre os países para várias especificações de modelagem diferentes. Concluímos com uma discussão de abordagens alternativas para modelar o crescimento do comércio eletrônico em um país. Os resultados também sugerem a adequação da exploração de modelos de contágio regional para o crescimento do comércio eletrônico.
        Abstract

Information and communication technologies (ICTs) continue to have a profound effect on the economies and societies where they are used. In this article, we propose three related theories to describe the underlying mechanism for growth in e-commerce revenues at the national level. Endogenous growth theory posits that the primary drivers of e-commerce growth are internal to a country. Exogenous growth theory suggests that the primary drivers of e-commerce growth are external to an economic system, and reflect the forces of the regional economy. A blend of these, a mixed endogenous–exogenous growth theory, incorporates drivers from both the economy and the region of a country. We test a number of hypotheses about e-commerce growth in the context of these theories. The key variables include Internet penetration, telecommunication investment intensity, venture capital and credit card availability, and education level. The data are drawn from 17 European countries over a five-year period from 2000 to 2004, and are analyzed using panel data regression with robust error terms, a variant of weighted least squares. The results show the differential efficacy of internal and external drivers as endogenous and exogenous precursors of e-commerce growth across the countries for a number of different modeling specifications. We conclude with a discussion of alternative approaches to model e-commerce growth in a country. The results also suggest the appropriateness of exploring models of regional contagion for e-commerce growth.

		% Lacuna de pesquisa - um bloco para cada lacuna
		\begin{lacuna}
		\label{lacuna:lacuna1}
			 "Estratégia de Marketing do Consumidor e E-Commerce na Última Década" oferece uma visão geral dos desenvolvimentos e tendências na estratégia de marketing do consumidor e no comércio eletrônico ao longo da última década.
		\end{lacuna}
	
		% Pergunta de pesquisa - um bloco para cada pergunta
		\begin{pergunta}
		\label{pergunta:pergunta_1}
			Aumento de citações sobre e-commerce entre 2010 e 2020 é de 80 por cento com pico de 245 citações em 2020.
		\end{pergunta}
  \begin{pergunta}
		\label{pergunta:pergunta_2}
			Pesquisa de palavras chaves  em publicações na Database SCOPUS, a palavra chave mais usada foi "consumidor" e a palavra "e-commerce" foi usado como filtro na publicação.
		\end{pergunta}

% Metodologia
\chapter{Metodologia}
\label{ch:identificador}
	\begin{resumocapitulo}
	O capítulo de metodologia na documentação descreve o processo estruturado e abrangente seguido no projeto de criação do e-commerce para a Tech Inova Shop. Iniciamos com um levantamento detalhado de requisitos, seguido por uma pesquisa de mercado e análise competitiva para entender as necessidades da empresa e as tendências do mercado. Selecionamos uma plataforma de e-commerce adequada e procedemos ao desenvolvimento e design do e-commerce, integrando funcionalidades essenciais e realizando testes rigorosos. Após a implementação e lançamento, monitoramos continuamente o desempenho do e-commerce e realizamos otimizações regulares para melhorar a experiência do usuário e maximizar as vendas. Este capítulo destaca a metodologia robusta utilizada no projeto, garantindo sua eficácia e sucesso.
	\end{resumocapitulo}

	\section{Visão Geral}
		O projeto de criação do e-commerce para a Tech Inova Shop foi conduzido seguindo uma metodologia estruturada e abrangente, que abordou cada etapa do processo de desenvolvimento com cuidado e precisão. Iniciamos o projeto com um extenso levantamento de requisitos, compreendendo as necessidades específicas da empresa e os objetivos de negócio almejados. Aprofundamos nossa compreensão do mercado de comércio eletrônico e da concorrência através de uma pesquisa detalhada, identificando tendências, oportunidades e desafios.
Com base nas informações coletadas, selecionamos a plataforma de e-commerce mais adequada para atender às demandas da Tech Inova Shop, considerando aspectos como funcionalidades, escalabilidade e custo-benefício. Em seguida, dedicamos esforços consideráveis ao desenvolvimento e design do e-commerce, garantindo não apenas uma interface atraente e intuitiva, mas também uma arquitetura robusta e segura.
A integração de funcionalidades essenciais, como sistema de pagamento, carrinho de compras e gerenciamento de estoque, foi realizada meticulosamente, seguida por testes rigorosos para garantir a qualidade e a segurança do sistema. Com todos os componentes do e-commerce prontos, procedemos à sua implementação e lançamento, seguindo as melhores práticas e padrões de comunicação organizacional.
Após o lançamento, iniciamos o monitoramento contínuo do desempenho do e-commerce, utilizando métricas de análise de dados para avaliar o sucesso e identificar áreas de melhoria. Realizamos otimizações regulares com base nos insights obtidos, visando melhorar a experiência do usuário e maximizar as vendas.
Com a implementação bem-sucedida do e-commerce para a Tech Inova Shop, a empresa agora desfruta de uma presença digital robusta e eficaz, oferecendo aos clientes uma plataforma de compras online conveniente e envolvente. O projeto não apenas cumpriu, mas excedeu as expectativas, proporcionando à empresa uma vantagem competitiva significativa no mercado de comércio eletrônico.

	\section{Conteúdo 1}
	\label{sec:identificao}
        METODOLOGIA

		% Lista numerada
		\begin{enumerate}
			\item Levantamento de Requisitos: Realizamos uma análise detalhada das necessidades e requisitos da Tech Inova Shop.
			\item Pesquisa de Mercado e Análise Competitiva:
Conduzimos uma pesquisa de mercado para entender as tendências do setor de comércio eletrônico e a concorrência.
            \item Definição da Plataforma de E-commerce:
Selecionamos a plataforma de e-commerce alinhada aos requisitos levantados.
             \item Desenvolvimento e Design do E-commerce:Desenvolvemos o e-commerce conforme os requisitos definidos.
             \item Integração de Funcionalidades e Testes:Integramos as funcionalidades essenciais e realizar testes de software para garantir a qualidade e a segurança do e-commerce.
             \item Implementamos e Lançamento:Realizar a implementação do e-commerce e planejar o lançamento.
             \item Monitoramento e Otimização:
Estabelecemos métricas de desempenho e realizar otimizações regulares para melhorar a experiência do usuário e maximizar as vendas.

		\end{enumerate}

		% Lacuna de pesquisa - um bloco para cada lacuna
		\begin{lacuna}
		\label{lacuna:lacuna1}
			Descrever aqui a lacuna de pesquisa. Se tiver mais que uma, criar outro bloco.
		\end{lacuna}
	
		% Pergunta de pesquisa - um bloco para cada pergunta
		\begin{pergunta}
		\label{pergunta:pergunta_1}
			Como a metodologia empregada na criação do e-commerce para a Tech Inova Shop influenciou a eficiência e o sucesso do projeto?
Resposta: A metodologia adotada proporcionou uma abordagem estruturada e abrangente, permitindo um levantamento detalhado de requisitos, uma análise precisa das necessidades da empresa e do mercado, e uma implementação cuidadosa do e-commerce. Isso resultou em uma plataforma robusta e eficaz que atendeu às expectativas da Tech Inova Shop e superou as demandas do mercado.
		\end{pergunta}

		\begin{pergunta}
		\label{pergunta:pergunta_2}
			Qual foi o impacto da metodologia na experiência do usuário e no desempenho das vendas do e-commerce da Tech Inova Shop?
Resposta: A metodologia aplicada priorizou a experiência do usuário, garantindo um design intuitivo, funcionalidades eficientes e segurança robusta. Isso contribuiu para uma experiência de compra agradável e sem problemas para os clientes, resultando em um aumento nas taxas de conversão e no volume de vendas para a Tech Inova Shop.
		\end{pergunta}	

% Resultado
\chapter{Análise dos Resultados}
\label{ch:resultados}
Após o desenvolvimento e implementação do e-commerce para a Tech Inova Shop, concluímos que o projeto foi um sucesso. A metodologia adotada proporcionou uma plataforma robusta e eficaz, atendendo às expectativas da empresa e oferecendo uma experiência de compra conveniente para os clientes. A priorização da experiência do usuário resultou em um design intuitivo, funcionalidades eficientes e segurança robusta, o que contribuiu para o aumento das taxas de conversão e do volume de vendas. Além disso, o monitoramento constante do desempenho do e-commerce permitiu identificar oportunidades de melhoria, garantindo uma melhoria contínua do sistema. Como resultado, a Tech Inova Shop conquistou uma vantagem competitiva significativa no mercado digital, expandindo sua presença e competindo de forma eficaz com seus concorrentes. Este projeto proporcionou valiosas lições aprendidas, que serão aplicadas em futuras iniciativas, contribuindo para o crescimento contínuo e sucesso da empresa no ambiente digital.
\section{Analise 1}
Sucesso da Implementação: O e-commerce foi implementado com sucesso, atendendo às necessidades e expectativas da Tech Inova Shop. A plataforma proporcionou uma experiência de compra conveniente e eficaz para os clientes, contribuindo para o crescimento dos negócios da empresa.
\section{Analise 2}
Impacto na Experiência do Usuário: A metodologia adotada priorizou a experiência do usuário, resultando em um design intuitivo, funcionalidades eficientes e segurança robusta. Isso contribuiu significativamente para a satisfação dos clientes e para o aumento das taxas de conversão.
\section{Analise 3}
Melhoria Contínua: O monitoramento constante do desempenho do e-commerce permitiu identificar áreas de melhoria e oportunidades de otimização. A empresa está comprometida com a melhoria contínua do sistema, visando sempre oferecer uma experiência ainda melhor aos clientes.
\section{Analise 4}
Vantagem Competitiva: O e-commerce da Tech Inova Shop proporcionou à empresa uma vantagem competitiva significativa no mercado, permitindo-lhe expandir sua presença digital e competir de forma eficaz com seus concorrentes.



% Conclusão
\chapter{Conclusões}
\label{ch:conclusao}
	Conclusão:

O projeto de criação do e-commerce para a Tech Inova Shop foi concluído com sucesso, representando um marco significativo na modernização e expansão dos negócios da empresa no mercado digital. A metodologia implementada permitiu o desenvolvimento de uma plataforma robusta e eficaz, que atendeu às necessidades da empresa e proporcionou uma experiência de compra conveniente e intuitiva para os clientes. A priorização da experiência do usuário resultou em um aumento nas taxas de conversão e no volume de vendas, fortalecendo a posição da Tech Inova Shop no mercado competitivo do comércio eletrônico.

Além disso, o monitoramento contínuo do desempenho do e-commerce possibilitou a identificação de oportunidades de melhoria, garantindo uma evolução constante do sistema. A empresa está comprometida com a melhoria contínua e a inovação, visando sempre oferecer a melhor experiência possível aos clientes e manter sua posição de destaque no mercado.

O sucesso deste projeto não apenas consolida a presença da Tech Inova Shop no ambiente digital, mas também representa um importante passo em direção ao futuro, onde a empresa continuará a explorar novas oportunidades e a buscar a excelência em todas as suas operações. Este projeto é apenas o começo de uma jornada de crescimento e sucesso contínuo da Tech Inova Shop no mercado global.

% ##################### Fim dos Capítulos ############################

% Bibliografia (Obrigatório)
\bibliography{refs}
Rosário,Albérico;Consumer Marketing Strategy and E-Commerce in the Last Decade: A Literature Review.(
GOVCOPP-Governance-1 November 2021);
Chun Ho-Shu - A growth theory perspective on B2C e-commerce growth in Europe: An exploratory study.Electronic Commerce Research and Applications
(Volume 6, Issue 3, Autumn 2007, Pages 237-259)

% ##################### Apêndices ############################
\renewcommand{\appendixtitle}{Apêndices}
\begin{appendixenv}
    \input{pos_textuais/01_apendice.tex}
\end{appendixenv}

% ##################### Anexos ############################
\renewcommand{\appendixtitle}{Anexos}
\begin{appendixenv}
    \section{: Título}
    Anexos
    \subsection*{Publicação}
        \begin{enumerate}
            \item Rosário,Albérico;~\textbf{Consumer Marketing Strategy and E-Commerce in the Last Decade: A Literature Review.(
GOVCOPP-Governance-1 November 2021) }. \textbf{In}: Chun Ho-Shu - A growth theory perspective on B2C e-commerce growth in Europe: An exploratory study.Electronic Commerce Research and Applications
(Volume 6, Issue 3, Autumn 2007, Pages 237-259)
        \end{enumerate}
\end{appendixenv}

\end{document}
