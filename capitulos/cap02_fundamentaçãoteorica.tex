\chapter{Fundamentação Teórica}
\label{ch:identificador}
	\begin{resumocapitulo}
		Aqui vai um pequeno resumo do capítulo.
	\end{resumocapitulo}

	\section{Visão Geral}
		Science Direct-Revista científica-ISSN 0346-251X
  Explorando os mecanismos  que impulsionam o crescimento do comércio eletrônico Business-to-Consumer (B2C) na Europa a partir de uma perspectiva teórica de crescimento.
  Exploring the mechanisms driving Business-to-Consumer (B2C) e-commerce growth in Europe from a growth theory perspective.
Explorando los mecanismos que impulsan el crecimiento del comercio electrónico Business-to-Consumer (B2C) en Europa desde una perspectiva teórica de crecimiento.
  

	\section{Conteúdo 1}
	\label{sec:identificao}
        Resumo
        As tecnologias de informação e comunicação (TICs) continuam a ter um profundo efeito nas economias e sociedades onde são utilizadas. Neste artigo, propomos três teorias relacionadas para descrever o mecanismo subjacente ao crescimento das receitas de comércio eletrônico em nível nacional. A teoria do crescimento endógeno postula que os principais impulsionadores do crescimento do comércio eletrônico são internos a um país. A teoria do crescimento exógeno sugere que os principais impulsionadores do crescimento do comércio eletrônico são externos a um sistema econômico e refletem as forças da economia regional. Uma combinação dessas teorias, uma teoria mista de crescimento endógeno-exógeno, incorpora impulsionadores tanto da economia quanto da região de um país. Testamos várias hipóteses sobre o crescimento do comércio eletrônico no contexto dessas teorias. As variáveis-chave incluem penetração da Internet, intensidade do investimento em telecomunicações, disponibilidade de capital de risco e cartão de crédito e nível de educação. Os dados são retirados de 17 países europeus ao longo de um período de cinco anos, de 2000 a 2004, e são analisados usando regressão de dados em painel com termos de erro robustos, uma variante de mínimos quadrados ponderados. Os resultados mostram a eficácia diferencial de impulsionadores internos e externos como precursores endógenos e exógenos do crescimento do comércio eletrônico entre os países para várias especificações de modelagem diferentes. Concluímos com uma discussão de abordagens alternativas para modelar o crescimento do comércio eletrônico em um país. Os resultados também sugerem a adequação da exploração de modelos de contágio regional para o crescimento do comércio eletrônico.
        Abstract

Information and communication technologies (ICTs) continue to have a profound effect on the economies and societies where they are used. In this article, we propose three related theories to describe the underlying mechanism for growth in e-commerce revenues at the national level. Endogenous growth theory posits that the primary drivers of e-commerce growth are internal to a country. Exogenous growth theory suggests that the primary drivers of e-commerce growth are external to an economic system, and reflect the forces of the regional economy. A blend of these, a mixed endogenous–exogenous growth theory, incorporates drivers from both the economy and the region of a country. We test a number of hypotheses about e-commerce growth in the context of these theories. The key variables include Internet penetration, telecommunication investment intensity, venture capital and credit card availability, and education level. The data are drawn from 17 European countries over a five-year period from 2000 to 2004, and are analyzed using panel data regression with robust error terms, a variant of weighted least squares. The results show the differential efficacy of internal and external drivers as endogenous and exogenous precursors of e-commerce growth across the countries for a number of different modeling specifications. We conclude with a discussion of alternative approaches to model e-commerce growth in a country. The results also suggest the appropriateness of exploring models of regional contagion for e-commerce growth.

		% Lacuna de pesquisa - um bloco para cada lacuna
		\begin{lacuna}
		\label{lacuna:lacuna1}
			 "Estratégia de Marketing do Consumidor e E-Commerce na Última Década" oferece uma visão geral dos desenvolvimentos e tendências na estratégia de marketing do consumidor e no comércio eletrônico ao longo da última década.
		\end{lacuna}
	
		% Pergunta de pesquisa - um bloco para cada pergunta
		\begin{pergunta}
		\label{pergunta:pergunta_1}
			Aumento de citações sobre e-commerce entre 2010 e 2020 é de 80 por cento com pico de 245 citações em 2020.
		\end{pergunta}
  \begin{pergunta}
		\label{pergunta:pergunta_2}
			Pesquisa de palavras chaves  em publicações na Database SCOPUS, a palavra chave mais usada foi "consumidor" e a palavra "e-commerce" foi usado como filtro na publicação.
		\end{pergunta}